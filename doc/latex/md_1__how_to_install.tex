\subsection*{Install from Source}

If you encounter some troubles, see \href{http://qulacs.org/md_4__trouble_shooting.html}{\texttt{ troubleshooting}}.

\subsubsection*{Requirements}


\begin{DoxyItemize}
\item python 2.\+7 or 3.\+x
\item gcc/g++ $>$= 7.\+0.\+0 or Visual\+Studio 2017
\item cmake $>$= 2.\+8
\item git
\end{DoxyItemize}

\subsubsection*{C++ Library(cppsim)}

\#\#\#\# G\+CC 
\begin{DoxyCode}{0}
\DoxyCodeLine{git clone https://github.com/qulacs/qulacs.git}
\DoxyCodeLine{cd qulacs}
\DoxyCodeLine{./script/build\_gcc.sh}
\end{DoxyCode}


\#\#\#\# M\+S\+VC 
\begin{DoxyCode}{0}
\DoxyCodeLine{git clone https://github.com/qulacs/qulacs.git}
\DoxyCodeLine{cd qulacs}
\DoxyCodeLine{generate\_msvc\_project.bat}
\end{DoxyCode}
 Then, open {\ttfamily Project.\+sln} in {\ttfamily ./qulacs/visualstudio/}, and build all.

\subsubsection*{Python Interface(\+Qulacs)}

Install 
\begin{DoxyCode}{0}
\DoxyCodeLine{git clone https://github.com/qulacs/qulacs.git}
\DoxyCodeLine{cd qulacs}
\DoxyCodeLine{python setup.py install}
\end{DoxyCode}


Uninstall 
\begin{DoxyCode}{0}
\DoxyCodeLine{pip uninstall qulacs}
\end{DoxyCode}


\subsection*{Gettig started}

See the following document for more detail.

\href{http://qulacs.org/md_2__tutorial__c_p_p.html}{\texttt{ C++ Tutorial}}

\href{http://qulacs.org/md_3__tutorial_python.html}{\texttt{ Python Tutorial}}

\href{https://github.com/qulacs/quantum-circuits}{\texttt{ Examples}}

\href{http://qulacs.org/annotated.html}{\texttt{ A\+PI document}}

\subsubsection*{C++ Libraries}

Add {\ttfamily ./$<$qulacs\+\_\+path$>$/include/} to include path, and {\ttfamily ./$<$qulacs\+\_\+path$>$/lib/} to library path. If you use dynamic link library, add {\ttfamily ./$<$qulacs\+\_\+path$>$/bin/} to library path instead.

Example of C++ code\+: 
\begin{DoxyCode}{0}
\DoxyCodeLine{\textcolor{preprocessor}{\#include <iostream>}}
\DoxyCodeLine{\textcolor{preprocessor}{\#include <cppsim/state.hpp>}}
\DoxyCodeLine{\textcolor{preprocessor}{\#include <cppsim/circuit.hpp>}}
\DoxyCodeLine{\textcolor{preprocessor}{\#include <cppsim/observable.hpp>}}
\DoxyCodeLine{}
\DoxyCodeLine{\textcolor{keywordtype}{int} main()\{}
\DoxyCodeLine{    QuantumState state(3);}
\DoxyCodeLine{    state.set\_Haar\_random\_state();}
\DoxyCodeLine{}
\DoxyCodeLine{    QuantumCircuit circuit(3);}
\DoxyCodeLine{    circuit.add\_X\_gate(0);}
\DoxyCodeLine{    \textcolor{keyword}{auto} merged\_gate = gate::merge(gate::CNOT(0,1),gate::Y(1));}
\DoxyCodeLine{    circuit.add\_gate(merged\_gate);}
\DoxyCodeLine{    circuit.add\_RX\_gate(1,0.5);}
\DoxyCodeLine{    circuit.update\_quantum\_state(\&state);}
\DoxyCodeLine{}
\DoxyCodeLine{    Observable observable(3);}
\DoxyCodeLine{    observable.add\_operator(2.0, \textcolor{stringliteral}{"X 2 Y 1 Z 0"});}
\DoxyCodeLine{    observable.add\_operator(-3.0, \textcolor{stringliteral}{"Z 2"});}
\DoxyCodeLine{    \textcolor{keyword}{auto} value = observable.get\_expectation\_value(\&state);}
\DoxyCodeLine{    std::cout << value << std::endl;}
\DoxyCodeLine{    \textcolor{keywordflow}{return} 0;}
\DoxyCodeLine{\}}
\end{DoxyCode}


Example of build command\+: 
\begin{DoxyCode}{0}
\DoxyCodeLine{g++ -I ./<qulacs\_path>/include -L ./<qulacs\_path>/lib <your\_code>.cpp -lcppsim.so}
\end{DoxyCode}


\subsubsection*{Python Libraries}

You can use features by simply importing {\ttfamily qulacs}.

Example of python code\+: 
\begin{DoxyCode}{0}
\DoxyCodeLine{from qulacs import Observable, QuantumCircuit, QuantumState}
\DoxyCodeLine{from qulacs.gate import Y,CNOT,merge}
\DoxyCodeLine{}
\DoxyCodeLine{state = QuantumState(3)}
\DoxyCodeLine{state.set\_Haar\_random\_state()}
\DoxyCodeLine{}
\DoxyCodeLine{circuit = QuantumCircuit(3)}
\DoxyCodeLine{circuit.add\_X\_gate(0)}
\DoxyCodeLine{merged\_gate = merge(CNOT(0,1),Y(1))}
\DoxyCodeLine{circuit.add\_gate(merged\_gate)}
\DoxyCodeLine{circuit.add\_RX\_gate(1,0.5)}
\DoxyCodeLine{circuit.update\_quantum\_state(state)}
\DoxyCodeLine{}
\DoxyCodeLine{observable = Observable(3)}
\DoxyCodeLine{observable.add\_operator(2.0, "X 2 Y 1 Z 0")}
\DoxyCodeLine{observable.add\_operator(-3.0, "Z 2")}
\DoxyCodeLine{value = observable.get\_expectation\_value(state)}
\DoxyCodeLine{print(value)}
\end{DoxyCode}
 